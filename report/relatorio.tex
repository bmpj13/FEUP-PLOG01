% This is LLNCS.DEM the demonstration file of
% the LaTeX macro package from Springer-Verlag
% for Lecture Notes in Computer Science,
% version 2.4 for LaTeX2e as of 16. April 2010
%
\documentclass{llncs}
%
\usepackage[portuguese]{babel}
\usepackage[utf8]{inputenc}
\usepackage{makeidx}  % allows for indexgeneration

%

\begin{document}

%
\frontmatter          % for the preliminaries
%
\pagestyle{headings}  % switches on printing of running heads
%
\title{Ano Escolar}
\subtitle{Resolução de Problemas de Otimização utilizando\\
Programação em Lógica com Restrições}
%
\titlerunning{Ano Escolar}  % abbreviated title (for running head)
%                                     also used for the TOC unless
%                                     \toctitle is used
%
\author{João Barbosa \and José Martins}
%
\authorrunning{João Barbosa \and José Martins} % abbreviated author list (for running head)
%
\institute{Faculdade de Engenharia da Universidade do Porto\\
Rua Roberto Frias, sn, 4200-465 Porto, Portugal}

\maketitle              % typeset the title of the contribution

\begin{abstract} %

Resolução de um problema de otimização referente a calendarização de testes e trabalhos de casa numa escola utilizando programação em lógica com restrições em SICStus Prolog , permitindo variar o numero de turmas, disciplinas, trabalhos de casa por dia e por disciplina assim como os horários de cada turma.
Foi implementado um conjunto de predicados de modo a aplicar todas as restrições pertinentes para uma boa resolução do problema em questão.

\textbf{
Deve contextualizar e resumir o trabalho, salientando o objetivo, o
método utilizado e fazendo referência aos principais resultados e à principal conclusão que
esses resultados permitem obter.}


\keywords{computational geometry, graph theory, Hamilton cycles}
\end{abstract}
%
\section{Introdução}
%
Este projeto esta a ser realizado no âmbito da unidade curricular Programação em lógica do Mestrado Integrado em Engenharia Informática e Computação.
O objetivo do projeto prende-se com a resolução de um problema de otimização utilizando programação em lógica com restrições.
O problema consiste na calendarização de testes e trabalhos de casa de um periodo letivo, com vários parametros váriaveis como o numero de turmas, disciplinas, trabalhos de casa (por dia e por disciplina) assim como a existencia de diferentes horários para cada turma, existem também diversas restrições que devem ser respeitadas.
Ao longo deste artigo será descrito o respetivo problema com um nivel de detalhe mais elevado, a abordagem levada a cabo pelo grupo para a implementação de uma possivel solução assim como as conclusões retiradas do solução implementada.

\textbf{
Descrição dos objetivos e motivação do trabalho, referência
sucinta ao problema em análise (idealmente, referência a outros trabalhos sobre o mesmo
problema e sua abordagem), e descrição sucinta da estrutura do resto do artigo.}

\newpage
\section{Descrição do Problema}
%

O Problema ''Ano Escolar'' é referente á maracação de testes e tpc ao longo do periodo sendo que os seguintes aspetos devem ser tidos em conta na resolução apresentada:

	\begin{itemize}
	\item Cada disciplina tem 2 testes por período de aulas, que decorrem num conjunto de semanas específico (mais ou menos a meioe no fim do período). 	
	\item Os alunos não podem ter mais do que 2 testes na mesma semana de aulas, nem testes em dias consecutivos.
	\item Em cada dia, não pode haver TPC em mais do que 2 disciplinas.
	\item Em pelo menos um dia por semana (que deve ser sempre o mesmo ao longo do
	  período), não pode haver TPC. 
	\item Em cada disciplina, só pode haver TPC em metade das aulas.
	\item Os testes realizados pelas diferentes turmas a uma
		mesma disciplina devem ser o mais próximos possível.
	\end{itemize}
O resultado deve incluir as datas dos testes de cada turma/disciplina, bem como os dias em que o professor de cada
disciplina pode mandar trabalho para casa.
Deve ser possível resolver problemas desta classe com diferentes parâmetros, como por exemplo variando o número de turmas e disciplinas, horários, número máximo de TPC por disciplina e por dia entre outros.


\textbf{
Descrever com detalhe o problema de otimização ou decisão em análise.}

\section{Abordagem}
%
A abordagem para o problema em questão consistiu inicialmente na avaliação das variáveis de decisão necessárias para a resolução do problema e por fim na melhor forma de as organizar numa estrutura que permitisse a colocação de restrições de um modo simples e eficiente.


\subsection{Variáveis de Decisão}
Descrever as variáveis de decisão e os seus domínios.

\subsection{Restrições}
 Descrever as restrições rígidas e flexíveis do problema e a sua implementação utilizando o SICStus Prolog.

\subsection{ Função de Avaliação}
Descrever, quando for o caso, a forma de avaliar a solução obtida e a sua implementação utilizando o SICStus Prolog.

\subsection{Estratégia de Pesquisa}
Descrever a estratégia de etiquetagem
(labeling) utilizada ou implementada, nomeadamente no que diz respeito à ordenação
de variáveis e valores.

\section{Visualização da Solução}
Explicar os predicados que permitem visualizar a solução em modo de texto

\section{Resultados}
Demonstrar exemplos de aplicação em instâncias do problema com
diferentes complexidades e analisar os resultados obtidos. Devem ser utilizadas formas
convenientes para apresentação dos resultados (tabelas e/ou gráficos).

\section{Conclusões e Trabalho Futuro}
 Que conclusões retira deste projeto? O que mostram os resultados obtidos? Quais as vantagens e limitações da
solução proposta? Como poderia melhorar o trabalho desenvolvido?

\begin{figure}
\vspace{2.5cm}
\caption{This is the caption of the figure displaying a white eagle and
a white horse on a snow field}
\end{figure}

\begin{table}
\caption{This is the example table taken out of {\it The
\TeX{}book,} p.\,246}
\begin{center}
\begin{tabular}{r@{\quad}rl}
\hline
\multicolumn{1}{l}{\rule{0pt}{12pt}
                   Year}&\multicolumn{2}{l}{World population}\\[2pt]
\hline\rule{0pt}{12pt}
8000 B.C.  &     5,000,000& \\
  50 A.D.  &   200,000,000& \\
1650 A.D.  &   500,000,000& \\
1945 A.D.  & 2,300,000,000& \\
1980 A.D.  & 4,400,000,000& \\[2pt]
\hline
\end{tabular}
\end{center}
\end{table}

\begin{equation}
\begin{array}{rcl}
  \dot{x}&=&JH' (x)\\
  x(0)&=&x (T)
\end{array}
\end{equation}


\begin{proposition}
Assume $H'(0)=0$ and $ H(0)=0$. Set:
\begin{equation}
  \delta := \liminf_{x\to 0} 2 N (x) \left\|x\right\|^{-2}\ .
  \label{eq:one}
\end{equation}

If $\gamma < - \lambda < \delta$,
the solution $\overline{u}$ is non-zero:
\begin{equation}
  \overline{x} (t) \ne 0\ \ \ \forall t\ .
\end{equation}
\end{proposition}


\paragraph{Notes and Comments.}
The results in this section are a
refined version of \cite{clar:eke};
the minimality result of Proposition
14 was the first of its kind.

To understand the nontriviality conditions, such as the one in formula
(\ref{eq:four}), one may think of a one-parameter family
$x_{T}$, $T\in \left(2\pi\omega^{-1}, 2\pi b_{\infty}^{-1}\right)$
of periodic solutions, $x_{T} (0) = x_{T} (T)$,
with $x_{T}$ going away to infinity when $T\to 2\pi \omega^{-1}$,
which is the period of the linearized system at 0.

%
% ---- Bibliography ----
%
\begin{thebibliography}{5}
%
\bibitem {url} 
SWI-Prolog,
\url{http://www.swi-prolog.org}
\bibitem {url} 
SICStus-Prolog,
\url{https://sicstus.sics.se}


\end{thebibliography}
\clearpage

\section*{Anexo}
\subsection*{Código fonte}

Bla Bla


\end{document}
