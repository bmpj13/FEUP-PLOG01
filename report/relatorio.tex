% This is LLNCS.DEM the demonstration file of
% the LaTeX macro package from Springer-Verlag
% for Lecture Notes in Computer Science,
% version 2.4 for LaTeX2e as of 16. April 2010
%
\documentclass{llncs}
%
\usepackage[portuguese]{babel}
\usepackage[utf8]{inputenc}
\usepackage{makeidx}  % allows for indexgeneration

%

\begin{document}

%
\frontmatter          % for the preliminaries
%
\pagestyle{headings}  % switches on printing of running heads
%
\title{Ano Escolar}
\subtitle{Resolução de Problemas de Otimização utilizando\\
Programação em Lógica com Restrições}
%
\titlerunning{Ano Escolar}  % abbreviated title (for running head)
%                                     also used for the TOC unless
%                                     \toctitle is used
%
\author{João Barbosa \and José Martins}
%
\authorrunning{João Barbosa \and José Martins} % abbreviated author list (for running head)
%
\institute{Faculdade de Engenharia da Universidade do Porto\\
Rua Roberto Frias, sn, 4200-465 Porto, Portugal}

\maketitle              % typeset the title of the contribution

\begin{abstract} %

Resolução de um problema de otimização referente a calendarização de testes e trabalhos de casa numa escola utilizando programação em lógica com restrições em SICStus Prolog , permitindo variar o numero de turmas, disciplinas, trabalhos de casa por dia e por disciplina assim como os horários de cada turma.
Foi implementado um conjunto de predicados de modo a aplicar todas as restrições pertinentes para uma boa resolução do problema em questão.

\textbf{
Deve contextualizar e resumir o trabalho, salientando o objetivo, o
método utilizado e fazendo referência aos principais resultados e à principal conclusão que
esses resultados permitem obter.}


\keywords{computational geometry, graph theory, Hamilton cycles}
\end{abstract}
%
\section{Introdução}
%
Este projeto esta a ser realizado no âmbito da unidade curricular Programação em lógica do Mestrado Integrado em Engenharia Informática e Computação.
O objetivo do projeto prende-se com a resolução de um problema de otimização utilizando programação em lógica com restrições.
O problema consiste na calendarização de testes e trabalhos de casa de um periodo letivo, com vários parametros váriaveis como o numero de turmas, disciplinas, trabalhos de casa (por dia e por disciplina) assim como a existencia de diferentes horários para cada turma, existem também diversas restrições que devem ser respeitadas.
Ao longo deste artigo será descrito o respetivo problema com um nivel de detalhe mais elevado, a abordagem levada a cabo pelo grupo para a implementação de uma possivel solução assim como as conclusões retiradas do solução implementada.

\textbf{
Descrição dos objetivos e motivação do trabalho, referência
sucinta ao problema em análise (idealmente, referência a outros trabalhos sobre o mesmo
problema e sua abordagem), e descrição sucinta da estrutura do resto do artigo.}

\newpage
\section{Descrição do Problema}
%

O Problema ''Ano Escolar'' é referente á maracação de testes e tpc ao longo do periodo sendo que as seguintes restrições devem ser tidas em conta na resolução apresentada:

	\begin{itemize}
	\item Cada disciplina tem 2 testes por período de aulas, que decorrem num conjunto de semanas específico (mais ou menos a meio e no fim do período). 	
	\item Os alunos não podem ter mais do que 2 testes na mesma semana de aulas, nem testes em dias consecutivos.
	\item Os testes realizados pelas diferentes turmas a uma
		    mesma disciplina devem ser o mais próximos possível.
	\item Em cada dia, não pode haver TPC em mais do que 2 disciplinas.
	\item Em pelo menos um dia por semana (que deve ser sempre o mesmo ao longo do
	  período), não pode haver TPC. 
	\item Em cada disciplina, só pode haver TPC em metade das aulas.
	\end{itemize}
O resultado deve incluir as datas dos testes de cada turma/disciplina, bem como os dias em que o professor de cada
disciplina pode mandar trabalho para casa.
Deve ser possível resolver problemas desta classe com diferentes parâmetros, como por exemplo variando o número de turmas e disciplinas, horários, número máximo de TPC por disciplina e por dia entre outros.


\textbf{
Descrever com detalhe o problema de otimização ou decisão em análise.}

\section{Abordagem}
%
A abordagem para o problema em questão consistiu inicialmente na avaliação das variáveis de decisão necessárias para a resolução do problema e por fim na elaboração de uma estrutura que permitisse a imposição de restrições de um modo simples e eficiente.
Depois decidiu-se de forma a separar conceitos elaborar predicados que resolvem o problema apenas para uma turma que depois são utilizados para resolver o problema geral.


\newpage
\subsection{Variáveis de Decisão}
A estrutura adoptada pelo grupo para resolver o problema consiste numa lista para cada turma com tamanho igual ao numero de disciplinas da turma, sendo que cada elemento dessa mesma lista é outra lista com 3 variaveis: 
\begin{itemize}
	\item Identificador da disciplina - Variavel já instanciada(conhecida pelo problema);
	\item Lista de testes - Lista com o tamanho igual ao numero de dias uteis do periodo, em que cada elemento da lista se trata de uma variavel de decisão com dominio de 0 a 1 simbolizando a existencia(1) ou não(0) de teste a disciplina em questão.
	\item Lista de trabalhos de casa - Lista com o tamanho igual ao numero de dias uteis do periodo, em que cada elemento da lista se trata de uma variavel de decisão com dominio de 0 a 1 simbolizando a existencia(1) ou não(0) de tabalhos de casa a disciplina em questão.
\end{itemize}

\textbf{
Descrever as variáveis de decisão e os seus domínios.}

\subsection{Restrições}
 Neste tópico serão abordadas as restrições do problema de otimização já mencionadas no ponto 2 com detalhes sobre a sua implementação. \\
 
\begin{enumerate}

	\item \textbf{Garantir dois testes por período para cada disciplina, mais ou menos a meio e no fim do período.}\\\\
		Para garantir o que o numero de testes por periodo para cada disciplina fosse igual a 2 e que cada teste estivesse numa época diferente (meio do periodo ou fim do periodo), foi utilizado o predicado 	               
		\textit{twoTestsPerPeriod(+Class)} que recebe uma turma com a estrutura especificada no ponto  \textbf{3.1} e para cada disciplina dessa turma utiliza a lista de variaveis de decisão que contem a informação dos 
		testes, obtendo depois duas sublistas a partir desta uma de 1/6  a 3/6  o numero de dias e outra de 4/6 a 6/6, que representam então as duas épocas de testes.
		Sendo depois garantida que em cada uma das épocas a soma dos elementos de cada lista é igual a 1 desta forma, exige-se a existencia de um teste em cada época de testes num total de dois testes por periodo.
		\\
	
\newpage
			
	\item \textbf{Máximo de 2 testes na mesma semana de aulas, e não permitir testes em dias consecutivos.} \\\\
		Para solucionar esta restrição foi utilizado o predicado \textit{testPlacementRestrictions(+Days,+Class)}  que recebe mais uma vez uma turma e numero de dias do periodo.
		Este predicado obtém numa primeira instancia uma lista que contém o numero total de testes em cada dia do periodo para a turma em questão (tendo em consideração todas as disciplinas), de seguida, verifica 
		com ajuda de predicados auxiliares que a soma dos testes em 2 dias seguidos tem que ser inferior ou igual a 1, impedindo assim a existencia de testes em dias seguidos e de vários testes no mesmo dia(foi tambem 
		tido em consideração os fins-de-semanas, sendo então possivel a existencia de testes na sexta-feira e na segunda-feira).
		\\

	\item \textbf{Os testes realizados pelas diferentes turmas a uma mesma disciplina devem ser o mais próximos possível.} \\\\
		De modo a permitir que os testes da mesma disciplina fossem o mais proximos possiveis em turmas diferentes foi utilizado o predicado  \textit{testsCloseBetweenClasses(+Classes, +DisciplineIds, -Sum1, -Sum2)} , 
		que recebe todas as turmas e os identificadores de todas as disciplinas lecionadas nas turmas em questão, retornando o somatório da diferença entre os testes da mesma disciplina em turmas diferentes para a 
		primeira época de testes(\textit{Sum1}) e para a segunda 
		(\textit{Sum2}), depois nas opções do labeling é feita a minimização destas variaveis utilizando \textit{minimize(Sum1) , minimize(Sum2)} na lista de opções do labeling.
		\\

	\item \textbf{Em cada dia, não pode haver TPC em mais do que N disciplinas.} \\\\
		A fim de permitir ao utilizador especificar o numero máximo de trabalhos de casa por dia, foi implementado o predicado \textit{maxNumberTpcPerDay(+Class,+Days,+N)} que em semelhança á restrição numero 2,
		obtem uma lista com o numero total de tpc's em cada dia do periodo para a turma em questão (tendo em consideração todas as disciplinas), sucedendo-se a seguir para cada elemento desta lista, uma comparação 
		de modo a garantir que o número de tpc's será inferior a N uma vez instanciadas as variaveis de decisão.
		\\

	\item \textbf{Em pelo menos um dia por semana (que deve ser sempre o mesmo ao longo do período), não pode haver TPC.} \\\\
		No sentido da implementação desta restrição foi criado o predicado \textit{clearTpcDay(Class,NoTpcDay)} que recebe o dia em que não existem trabalhos de casa e depois percorre a lista de disciplinas da turma 
		de forma a que na lista com as variáveis de decisão que dizem respeito aos tpc's, sejam instanciadas(com 0) de forma a indicar a não existencia de tpc's todas as variaveis de decisão que dizem respeito ao dia da 
		semana especificado nos parâmetros do predicado.
		\\

	\item \textbf{Em cada disciplina, só pode haver TPC numa percentagem das aulas.} \\\\
		Em relação a esta restrição foi concebido o predicado  \textit{limitNumberOfTpcPerPeriod(+Class,+Ratio,+Schedule,+Days)} que recebe a razão especificada pelo utilizador (ex: 2 - metade, 3 - um terço, 4 - um 
		quarto ) e numa primeira fase é obtido o numero total de aulas de uma disciplina ao longo do período sendo que depois a soma das variaveis de decisão referentes aos trabalhos de casa é limitada de modo a ser 
		inferior ou igual a  1/\textit{Ratio}, este processo é de seguida repetido para todas as disciplinas da turma. 
	 	\\

	\end{enumerate}
		
	
	
 
 

\subsection{ Função de Avaliação}
Descrever, quando for o caso, a forma de avaliar a solução obtida e a sua implementação utilizando o SICStus Prolog.

\subsection{Estratégia de Pesquisa}

A estratégia de etiquetagem passa por utilizar o predicado de \textit{labeling} com a seguinte estrutura:\newline
\centerline{\textit{labeling([ff, down, minimize(Sum1), minimize(Sum2)], R)}}\newline
 
Em que
\begin{itemize}
\item Sum1 - Soma das somas entre a diferença de dias para cada disciplina, entre cada turma, para a primeira onda de testes.
\item Sum2 - Soma das somas entre a diferença de dias para cada disciplina, entre cada turma, para a segunda onda de testes.
\item R - lista de variáveis de domínio, a serem instanciadas.
\end{itemize}
 
A opção \textit{\textbf{ff}} melhora a eficiência do programa, utilizando a estratégia de \textit{First-Fail}.\newline
Utilizando \textit{\textbf{down}}, a instanciação das variáveis ocorre por ordem descendente do seu intervalo de domínio. Esta opção permite, de forma simples e eficiente, garantir que são colocados o número máximo possível de TPCs para as turmas, que obedeçam às restrições dadas. Note-se que outra solução seria usar a opção \textit{maximize}, para a soma de todos os TPCs, mas essa estratégia seria mais lenta, oferecendo os mesmos resultados.\newline
Finalmente, o \textit{minimize} para as variáveis \textit{Sum1} e \textit{Sum2} garante que os testes, entre turmas, estão o mais próximos entre si, o quanto possível.
 
\textbf{
Descrever a estratégia de etiquetagem
(labeling) utilizada ou implementada, nomeadamente no que diz respeito à ordenação
de variáveis e valores.}


\section{Visualização da Solução}
Explicar os predicados que permitem visualizar a solução em modo de texto

\section{Resultados}
Demonstrar exemplos de aplicação em instâncias do problema com
diferentes complexidades e analisar os resultados obtidos. Devem ser utilizadas formas
convenientes para apresentação dos resultados (tabelas e/ou gráficos).

\section{Conclusões e Trabalho Futuro}
 Que conclusões retira deste projeto? O que mostram os resultados obtidos? Quais as vantagens e limitações da
solução proposta? Como poderia melhorar o trabalho desenvolvido?

\begin{figure}
\vspace{2.5cm}
\caption{This is the caption of the figure displaying a white eagle and
a white horse on a snow field}
\end{figure}

\begin{table}
\caption{This is the example table taken out of {\it The
\TeX{}book,} p.\,246}
\begin{center}
\begin{tabular}{r@{\quad}rl}
\hline
\multicolumn{1}{l}{\rule{0pt}{12pt}
                   Year}&\multicolumn{2}{l}{World population}\\[2pt]
\hline\rule{0pt}{12pt}
8000 B.C.  &     5,000,000& \\
  50 A.D.  &   200,000,000& \\
1650 A.D.  &   500,000,000& \\
1945 A.D.  & 2,300,000,000& \\
1980 A.D.  & 4,400,000,000& \\[2pt]
\hline
\end{tabular}
\end{center}
\end{table}

\begin{equation}
\begin{array}{rcl}
  \dot{x}&=&JH' (x)\\
  x(0)&=&x (T)
\end{array}
\end{equation}


\begin{proposition}
Assume $H'(0)=0$ and $ H(0)=0$. Set:
\begin{equation}
  \delta := \liminf_{x\to 0} 2 N (x) \left\|x\right\|^{-2}\ .
  \label{eq:one}
\end{equation}

If $\gamma < - \lambda < \delta$,
the solution $\overline{u}$ is non-zero:
\begin{equation}
  \overline{x} (t) \ne 0\ \ \ \forall t\ .
\end{equation}
\end{proposition}


\paragraph{Notes and Comments.}
The results in this section are a
refined version of \cite{clar:eke};
the minimality result of Proposition
14 was the first of its kind.

To understand the nontriviality conditions, such as the one in formula
(\ref{eq:four}), one may think of a one-parameter family
$x_{T}$, $T\in \left(2\pi\omega^{-1}, 2\pi b_{\infty}^{-1}\right)$
of periodic solutions, $x_{T} (0) = x_{T} (T)$,
with $x_{T}$ going away to infinity when $T\to 2\pi \omega^{-1}$,
which is the period of the linearized system at 0.

%
% ---- Bibliography ----
%
\begin{thebibliography}{5}
%
\bibitem {url} 
SWI-Prolog,
\url{http://www.swi-prolog.org}
\bibitem {url} 
SICStus-Prolog,
\url{https://sicstus.sics.se}


\end{thebibliography}
\clearpage

\section*{Anexo}
\subsection*{Código fonte}

Bla Bla


\end{document}
