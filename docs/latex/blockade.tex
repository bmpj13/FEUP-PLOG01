\documentclass[a4paper]{article}

%use the english line for english reports
%usepackage[english]{babel}
\usepackage[portuguese]{babel}
\usepackage[utf8]{inputenc}
\usepackage{indentfirst}
\usepackage{graphicx}
\usepackage{verbatim}
\usepackage{fancyhdr}


\pagestyle{fancy}
\fancyhf{}
\renewcommand{\headrulewidth}{0pt}
\fancyhead{}
\fancyfoot[R]{\thepage}

\graphicspath{{./images/}}


\begin{document}
\nocite{*}

\setlength{\textwidth}{16cm}
\setlength{\textheight}{22cm}

\title{\Huge\textbf{Blockade}\linebreak\linebreak\linebreak
\Large\textbf{Relatório Final}\linebreak\linebreak
\linebreak\linebreak
\includegraphics[scale=0.1]{feup-logo.png}\linebreak\linebreak
\linebreak\linebreak
\Large{Mestrado Integrado em Engenharia Informática e Computação} \linebreak\linebreak
\Large{Programação em Lógica}\linebreak
}

\author{\textbf{Grupo Blockade-4:}\\  João Barbosa - up201406241 \\ José Martins - up201404189 \\\linebreak\linebreak \\
 \\ Faculdade de Engenharia da Universidade do Porto \\ Rua Roberto Frias, s\/n, 4200-465 Porto, Portugal 
\vspace{1cm}}
\date{12 de Novembro de 2016}
\maketitle
\thispagestyle{empty}

%************************************************************************************************
%************************************************************************************************

\newpage

\section*{Resumo}
No âmbito da unidade curricular "Programação em Lógica" do 3º ano do Mestrado Integrado em Engenharia Informática e Computação, da Faculdade de Engenharia da Universidade do Porto, foi proposto o desenvolvimento de um jogo de tabuleiro, utilizando a linguagem \textit{Prolog}. O jogo  escolhido denomina-se \textbf{Blockade}. \par
A utilização de \textit{Prolog}, uma linguagem do paradigma da programação lógica, mostrou-se não só interessante, pois foi possível perceber as grandes potencialidades e vantagens da linguagem, como também desafiante, devido à inexperiência neste tipo de paradigma. \par
Para o desenvolvimento do projeto, foi necessário perceber e estruturar os conceitos-chave do jogo, de forma a obter a melhor forma de proceder à implementação do jogo. Além disso, foi necessário aprofundar o conhecimento sobre as bibliotecas disponibilizadas pela linguagem, para garantir qualidade e segurança no código. \par
O resultado final da implementação do jogo mostra-se como uma boa reprodução do \textbf{Blockade}. Adicionalmente, permite que os jogadores sejam \textbf{humanos} (que escolhem as jogadas que pretendem fazer), ou \textbf{\textit{bots}} (que avaliam a melhor jogada que podem executar).

\newpage

\tableofcontents

%************************************************************************************************
%************************************************************************************************

%*************************************************************************************************
%************************************************************************************************

\newpage

%%%%%%%%%%%%%%%%%%%%%%%%%%
\section{Introdução}
Este projeto foi desenvolvido no âmbito da unidade curricular “Programação em Lógica” do 3º ano do Mestrado Integrado em Engenharia Informática e Computação, da Faculdade de Engenharia da Universidade do Porto. O seu objetivo é implementar em \textit{Prolog} um jogo de tabuleiro de 2 jogadores, possibilitando os modos de jogo \texttt{Humano vs. Humano, Humano vs. Computador e Computador vs. Computador}. \par
Neste relatório será descrito o jogo que escolhemos para a nossa implementação – o \textbf{Blockade} – assim como as suas regras. De seguida, serão detalhadas algumas funcionalidades e características da nossa implementação, desde a representação do jogo e visualização do tabuleiro, até à avaliação de jogadas pelo computador e final do jogo. \par
Por fim, serão apresentadas as conclusões que obtivemos da realização deste trabalho, bem como a sua bibliografia.



%%%%%%%%%%%%%%%%%%%%%%%%%%
\newpage
\section{O Jogo Blockade}

Blockade trata-se de um jogo de tabuleiro, produzido pela primeira vez em 1975 pela Lakeside Games.
O jogo é desenhado para 2 jogadores sendo que cada um possui: 

\begin{itemize}
\item 2 Peões
\item 9 Parede verdes (que só podem ser colocadas verticalmente)
\item 9 Paredes azuis (que só podem ser colocadas horizontalmente)
\end{itemize}

O tabuleiro do jogo é um quadriculado com dimensões 11x14, com 2 pontos amarelos e dois pontos laranja que representam as bases, e as posições iniciais, dos dois peões de cada jogador. Estes pontos distam 4 quadrículas de cada canto na diagonal. 

\begin{center}
\includegraphics[scale = 0.3]{fig1.jpg}
\end{center}

Trata-se de um jogo de turnos, em que em cada turno um jogador pode mover um dos seus peões, uma ou duas quadrículas (horizontalmente, verticalmente ou uma combinação das duas), e posicionar uma parede de modo a tentar bloquear os movimentos do adversário. As paredes ocupam sempre duas quadrículas e devem ser posicionadas de acordo com a sua cor. Peões podem saltar por cima de outros peões que estejam a bloquear o seu caminho. \par
O objetivo do jogo é levar um dos seus peões até à base de um dos peões do adversário. Quando os jogadores ficarem sem paredes para colocar, continuam a mover-se até que alguém vença o jogo. 

\begin{center}
\includegraphics[scale = 0.4]{fig2.jpg}
\end{center}

%%%%%%%%%%%%%%%%%%%%%%%%%%
\section{Lógica do Jogo}


\subsection{Representação do Estado do Jogo} 

O jogo possui uma representação interna, utilizada para o processamento e armazenamento de informação, e uma representação externa, para tornar a visualização do jogo mais apelativa e intuitiva. A simbologia utilizada é a seguinte: 


\begin{center}
\includegraphics[scale = 0.64]{fig3.png}
\end{center}

O tabuleiro é tambem representado internamente por um grafo, de modo a permitir a utilização de algoritmos de \textit{pathfinding} na avaliação das melhores jogadas, e em algumas verificações.

%%%%%%%%%%%%%%%%%%%%%%%%%%
\newpage
\subsection{Visualização do Tabuleiro} 
O tabuleiro será visualizado através da utilização de caracteres ASCII para representar os peãos, paredes e bases de cada jogagor, exemplo:


\begin{center}
\includegraphics[scale = 0.68]{fig4.png}
\end{center}


\subsection{Lista de Jogadas Válidas} 

As jogadas são obtidas através do \textit{input} do jogador, ou através dos algoritmos implementados para permitir o cálculo da jogada do computador, sendo posteriormente verificada a sua validade. \par
Note-se que, por norma, não é necessária a utilização uma \textbf{lista} de jogadas válidas, devido à natureza do jogo. Os predicados que retornam listas de possibilidades são:
\begin{itemize}
	\item  \textit{evaluateBestDirection(+Player,+Id,-Directions)}
	\item 	\textit{evaluateBestWall(+Player,-Walls)}
\end{itemize}

%%%%%%%%%%%%%%%%%%%%%%%%%%
\newpage
\subsection{Execução de Jogadas} 

Depois de obtidas as coordenadas para a movimentação do jogador é utilizado o predicado \textit{validPosition(+Pawn,+ Board, +X,+ Y, -Nx, -Ny)}. Este predicado recebe o offset (X,Y) para onde o jogador se quer mover em relação à sua posição atual, e falha quando as coordenadas finais da ''futura'' posição do jogador se encontram fora das dimensões do tabuleiro, ou quando existe uma parede a bloquear a movimentação para as novas coordenadas.
\par Depois de a jogada ser validada, o predicado \textit{moveOneSpace(+Pawn, +X, +Y, +Board, -NewBoard)} move o peão num offset (X,Y), criando um novo tabuleiro.
\par Existe tambem outro predicado para validar e posicionar as paredes, denominado \textit{placeWall(+Player,+X,+ Y,+O,+Board, -NewBoard)}. O predicado é bem sucedido quando as coordenadas da parede são validas, i.e. quando o seguinte não acontece:
\begin{itemize}
	\item A parede está para lá dos limites do tabuleiro
	\item A parede está cruzada com outra parede
	\item A parede bloqueia completamente o jogador (ou seja quando o posicionamento da parede impossibilita que um dos peões deixe de ter caminho para as bases adversárias)
\end{itemize}
Se \textit{placeWall} for invocado com sucesso, cria um novo tabuleiro com as informações atualizadas. Além disso, este predicado vai também atualizar o grafo do tabuleiro.



\subsection{Avaliação do Tabuleiro} Na nossa implementação, não existe uma avaliação direta do tabuleiro, para comparar as diferentes jogadas possíveis, pois:
\begin{itemize}
	\item Quando o jogador é um humano, a direção do movimento é pedida via linha de comandos. Neste contexto, a avaliação da jogada é ignorada, pois o jogador deve ter liberdade para a fazer, independentemente da sua qualidade.
	
	\item Quando o jogador é um computador em nível fácil, os movimentos são aleatórios, não havendo qualquer avaliação do tabuleiro.
	
	\item Quando o jogador é um computador em nível difícil, a movimentação das peças é feita com recurso a \textit{pathfinding}, que nos devolve a melhor jogada possível. \\
Este algoritmo avalia internamente o grafo, e determina qual o caminho de menor custo para o jogador. Esta avaliação do tabuleiro é indireta, visto que os algoritmos sobre grafos utilizados estão implementados nas biliotecas do \textit{Prolog}.
\end{itemize}


%%%%%%%%%%%%%%%%%%%%%%%%%%
\newpage
\subsection{Final do Jogo} A verificação de que o jogo chegou ao fim é testada usando o predicado \mbox{\textit{checkEnd}}. Este predicado não precisa de argumentos, pois as posições dos jogadores são extraídas diretamente da base de dados, utilizando o predicado auxiliar \textit{position(+Player, -X, -Y)}. \par
Se algum dos jogadores chegar ao seu objetivo, i.e. a uma das bases do oponente, então o predicado retorna \textit{true}, e, consequentemente, termina o ciclo de jogo.




\subsection{Jogada do Computador} 

Foram implementados dois niveis de dificuldade para a jogada do computador:

\textbf{Nivel Fácil}
\begin{itemize}
\item Este modo é totalmente aleatório, e não realiza nenhuma consideração sobre o estado de jogo. Para criar a jogada são utilizados os predicados:
	\begin{itemize}
		\item \textit{randomMove(-X, -Y)} - para calcular uma movimentação aleatória.
		\item \textit{randomWall(-X, -Y, -O)} - para calcular uma parede aleatória.
	\end{itemize}
	 Quando as coordenadas devolvidas não são validas, os predicados voltam a ser executados, até ser encontrada uma jogada possivel.
\end{itemize}

\textbf{Nivel Difícil}
\begin{itemize}
\item Este modo procura a melhor jogada possivel no momento, utilizando \textit{pathfinding} para encontrar o caminho mais curto, tendo em consideração todas as paredes já posicionadas. Assim, é escolhido de entre os dois peões do jogador o melhor a ser movido, e a melhor direção para o mover.
\par No que diz respeito às paredes, é feito o processo contrário ou seja, o predicado determina o melhor peão do oponente e a melhor direção para ele se movimentar. 
\par Utilizando estas informações, o computador calcula a lista de jogadas que melhor bloqueiam o movimento do oponente, ordenadas por ordem de crescente de qualidade. Se uma parede não pode ser colocada, é escolhida a seguinte, num processo recursivo. Se nenhuma parede pode ser colocada, então o computador decide não colocar nenhuma parede nessa jogada.

 Para criar a jogada neste nível são utilizados os predicados:
 \begin{itemize}
		\item \textit{evaluateBestPawn(+Player,-N)} - para calcular o melhor peão a movimentar.
		\item \textit{evaluateBestDirectionPro(+Player,+Id, -Direction)}-  para calcular a melhor direção para o peão.
		\item \textit{evaluateBestWall(+Player,-Walls)}- para calcular uma lista ordenada com as melhores paredes a posicionar para bloquear o oponente de Player.
	\end{itemize}
\end{itemize}







%%%%%%%%%%%%%%%%%%%%%%%%%%
\newpage
\section{Interface com o Utilizador}


Foram criados diversos menus para permitir uma fácil interação do utilizador com o programa.
\par As opções possiveis estão sempre explicitas nos menus, o utilizador deverá colocar um '.' a seguir á opção pretendida  (ex. '1.').
\par Aquando da chamada do predicado \textit{blockade} (sem argumentos) o jogo iniciasse, sendo mostrado ao utilizador o seguinte menu.

\begin{center}
	\includegraphics[scale = 0.7]{fig5.png}
\end{center}

Se a escolha do utilizador passar pela opção 2 ou 3 é apresentado o seguinte menu, para permitir a seleção da diculdade pretendida.

\begin{center}
	\includegraphics[scale = 0.7]{fig6.png}
\end{center}

Em ambiente de jogo é representado o tabuleiro atual e no fim deste são apresentados os dialogos correspondentes á ação a ser tomada no momento.

  \begin{itemize}
		\item Diálogo para a movimentação de um peão do jogador.	
		\begin{center}
			\includegraphics[scale = 0.7]{fig7.png}
		\end{center}
		\item Diálogo para a colocação de uma parede.
		\begin{center}
			\includegraphics[scale = 0.7]{fig8.png}
		\end{center}
	\end{itemize}






%%%%%%%%%%%%%%%%%%%%%%%%%%
\section{Conclusões}
Tendo em conta resultado final da implementação, é possível afirmar que os principais objetivos deste projeto foram cumpridos:
\begin{itemize}
	\item Reprodução completa do jogo \textbf{Blockade}.
	\item Modos de jogo \texttt{Humano vs Humano}, \texttt{Humano vs Computador}, \texttt{Computador vs Computador}.
	\item Dois níveis de inteligência para o \textbf{\textit{bot}}.
	\item Interface com o utilizador
\end{itemize}
\ \par
Durante o desenvolvimento, foi possível perceber as potencialidades do \textit{Prolog}, e as vantagens de utilizar programação em lógica: a codificação das operações de natureza puramente lógica tornam-se mais intuitivas de realizar, e a implementação de certos algoritmos é facilitada pela natureza da linguagem. \\
É também de notar que o \textit{Prolog} possui muitas bibliotecas que podem ajudar a construir código com qualidade e segurança e, por esse motivo, foram utilizadas sempre que fossem convenientes. \par
No entanto, é importante reconhecer que, devido à inexperiência na linguagem, algumas implementações poderiam ser melhoradas. Especificamente, o tabuleiro de jogo possui 2 representações internas: uma em lista de listas, e outra em grafo. Isto acontece porque a representação inicial utilizada foi lista de listas, mas devido ao \textit{pathfinding}, foi também necessário a criação de um grafo. \par
As duas representações foram mantidas porque:
\begin{itemize}
	\item A maioria dos predicados utiliza a representação em lista de listas.
	\item O \textit{refactor} dos mesmos predicados seria custoso, em termos de tempo.
	\item A utilização única de representação em grafo provocaria algum \textit{overhead} em certas componentes (p.ex no \textit{display} do tabuleiro).
\end{itemize}
No entanto, esta solução apresenta algumas desvantagens:
\begin{itemize}
	\item Ocupa mais espaço em memória.
	\item A atualização do tabuleiro ocorre em dois sítios.
\end{itemize}
Um maior conhecimento prévio do \textit{Prolog}, e das bibliotecas que oferece, preveniria este acontecimento. \par
\ \par
Concluímos, portanto, que este projeto possibilitou a aprendizagem e consolidação de um novo tipo de paradigma de programação, abrindo novos horizontes e novas possibilidades, na definição e implementação de soluções para novos problemas.


\clearpage
\addcontentsline{toc}{section}{Bibliografia}
\renewcommand\refname{Bibliografia}
\bibliographystyle{plain}
\bibliography{myrefs}

\newpage
\appendix
\section{Anexo Código}
O código encontra-se disponível no ficheiro \textit{blockade.zip}.

\end{document}
